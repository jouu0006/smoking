% Options for packages loaded elsewhere
\PassOptionsToPackage{unicode}{hyperref}
\PassOptionsToPackage{hyphens}{url}
%
\documentclass[
]{article}
\usepackage{amsmath,amssymb}
\usepackage{lmodern}
\usepackage{setspace}
\usepackage{iftex}
\ifPDFTeX
  \usepackage[T1]{fontenc}
  \usepackage[utf8]{inputenc}
  \usepackage{textcomp} % provide euro and other symbols
\else % if luatex or xetex
  \usepackage{unicode-math}
  \defaultfontfeatures{Scale=MatchLowercase}
  \defaultfontfeatures[\rmfamily]{Ligatures=TeX,Scale=1}
\fi
% Use upquote if available, for straight quotes in verbatim environments
\IfFileExists{upquote.sty}{\usepackage{upquote}}{}
\IfFileExists{microtype.sty}{% use microtype if available
  \usepackage[]{microtype}
  \UseMicrotypeSet[protrusion]{basicmath} % disable protrusion for tt fonts
}{}
\makeatletter
\@ifundefined{KOMAClassName}{% if non-KOMA class
  \IfFileExists{parskip.sty}{%
    \usepackage{parskip}
  }{% else
    \setlength{\parindent}{0pt}
    \setlength{\parskip}{6pt plus 2pt minus 1pt}}
}{% if KOMA class
  \KOMAoptions{parskip=half}}
\makeatother
\usepackage{xcolor}
\usepackage[margin=1in]{geometry}
\usepackage{longtable,booktabs,array}
\usepackage{calc} % for calculating minipage widths
% Correct order of tables after \paragraph or \subparagraph
\usepackage{etoolbox}
\makeatletter
\patchcmd\longtable{\par}{\if@noskipsec\mbox{}\fi\par}{}{}
\makeatother
% Allow footnotes in longtable head/foot
\IfFileExists{footnotehyper.sty}{\usepackage{footnotehyper}}{\usepackage{footnote}}
\makesavenoteenv{longtable}
\usepackage{graphicx}
\makeatletter
\def\maxwidth{\ifdim\Gin@nat@width>\linewidth\linewidth\else\Gin@nat@width\fi}
\def\maxheight{\ifdim\Gin@nat@height>\textheight\textheight\else\Gin@nat@height\fi}
\makeatother
% Scale images if necessary, so that they will not overflow the page
% margins by default, and it is still possible to overwrite the defaults
% using explicit options in \includegraphics[width, height, ...]{}
\setkeys{Gin}{width=\maxwidth,height=\maxheight,keepaspectratio}
% Set default figure placement to htbp
\makeatletter
\def\fps@figure{htbp}
\makeatother
\setlength{\emergencystretch}{3em} % prevent overfull lines
\providecommand{\tightlist}{%
  \setlength{\itemsep}{0pt}\setlength{\parskip}{0pt}}
\setcounter{secnumdepth}{-\maxdimen} % remove section numbering
\ifLuaTeX
  \usepackage{selnolig}  % disable illegal ligatures
\fi
\IfFileExists{bookmark.sty}{\usepackage{bookmark}}{\usepackage{hyperref}}
\IfFileExists{xurl.sty}{\usepackage{xurl}}{} % add URL line breaks if available
\urlstyle{same} % disable monospaced font for URLs
\hypersetup{
  pdftitle={Death rate from smoking},
  hidelinks,
  pdfcreator={LaTeX via pandoc}}

\title{Death rate from smoking}
\author{true}
\date{}

\begin{document}
\maketitle

\setstretch{1.5}
\hypertarget{introduction}{%
\subsubsection{Introduction}\label{introduction}}

Smoking is one of the world's biggest health problems. Millions of
people suffer from poor health as a result of smoking, and researchers
estimate that smoking causes about 8 million premature deaths each year.

\textbf{Q1: Which two countries had the highest smoking death rates in
2019?}

\textbf{Q2: What are the trends in these two countries over the past
decade?}

\textbf{(1).The table of top two countries}

\begin{longtable}[]{@{}lr@{}}
\caption{The two countries with the highest death rates from
smoking}\tabularnewline
\toprule()
Country & rate \\
\midrule()
\endfirsthead
\toprule()
Country & rate \\
\midrule()
\endhead
Kiribati & 0.3679994 \\
Solomon Islands & 0.3287783 \\
\bottomrule()
\end{longtable}

From the table of The two countries with the highest death rates from
smoking we can see that at the top of the list is Kiribati with almost
0.37\% death rate. And the following country is Solomon Islands which
death rate is 0.33\%.

\textbf{(2).The figure of trend of death rate in Kiribati and Solomon
Islands}

\begin{figure}
\centering
\includegraphics{ETC5513_files/figure-latex/KiribatiAndSolomonfigure-1.pdf}
\caption{Death rate in Kiribati and Solomon}
\end{figure}

The graph of the trend of death rate in Kiribati and Solomon shows that
in the last decade, the number of deaths caused by smoking in Kiribati
has been decreasing year by year. In Solomon Islands, the number
fluctuated between 2009 to 2019 and has been on the rise in recent
years.

\hypertarget{summary}{%
\subsubsection{Summary}\label{summary}}

To this dataset, I selected the data of the world of death rate from
smoking from 2009 to do a research. The data means the annual number of
deaths attributed to smoking per 100,000 people. My conclusion is
Although the number of deaths from smoking is decreasing in some
countries, we still need to pay attention to the health risks of
smoking. Smoking is not only harmful to your own health, but also
affects the health of those around you. Therefore, we should appeal the
public to stop smoking.

\end{document}
